\chapter*[Introdução]{Introdução}
\addcontentsline{toc}{chapter}{Introdução}

A moldagem por injeção é um dos processos mais amplamente utilizados na fabricação de produtos plásticos, devido à sua eficiência em produzir grandes volumes de peças com baixo custo e alta qualidade. Este trabalho de conclusão de curso tem como objetivo principal o estudo e a implementação de moldes para máquinas injetoras, especificamente a Injetora Romi Prática 80, presente no laboratório da universidade. A partir de uma revisão bibliográfica, abordaremos os fundamentos do processo de injeção de plásticos e os princípios de projeto de moldes, explorando os materiais comumente utilizados nesse processo e as etapas de fabricação do produto final.

O foco do projeto será a criação de corpos de prova, que serão utilizados em ensaios de tração e outros testes laboratoriais, fornecendo uma aplicação prática para a máquina injetora da universidade. Para apoiar o desenvolvimento do molde, os softwares Moldflow e Ansys serão empregados para analisar o comportamento dos fluidos durante a injeção, garantindo precisão e eficiência no processo de fabricação. Além de atender às necessidades internas da universidade, este trabalho busca criar um modelo de molde que possa ser replicado por futuros estudantes e profissionais, facilitando a criação de novos moldes para diversas aplicações.

A relevância deste trabalho se destaca tanto no contexto acadêmico quanto industrial, uma vez que o método de fabricação por injeção é amplamente utilizado pela sua capacidade de produzir peças complexas de forma rápida e econômica. Ao final deste projeto, espera-se que a máquina injetora seja mais amplamente utilizada para a produção de corpos de prova, além de fornecer uma base sólida para futuros estudos e projetos relacionados ao design de moldes para injeção de plásticos.



