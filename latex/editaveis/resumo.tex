\begin{resumo}
    Este trabalho apresenta um estudo detalhado sobre o desenvolvimento de moldes para máquinas de injeção de plásticos, com foco no processo de fabricação e no projeto de moldes. A pesquisa foi baseada na utilização da máquina injetora Romi Prática 80, disponível na universidade, com o objetivo de projetar e fabricar moldes para a criação de corpos de prova utilizados em testes laboratoriais. A metodologia inclui uma revisão bibliográfica sobre o processo de injeção, materiais adequados e as etapas envolvidas na fabricação de produtos injetados. Além disso, foram empregados os softwares Moldflow e Ansys para análise do comportamento dos fluidos durante a injeção. O projeto do molde visa, além da fabricação de corpos de prova, fornecer um modelo de referência que possa ser replicado para a criação de outros tipos de moldes. A importância deste trabalho reside na sua contribuição para o uso acadêmico da máquina injetora e no potencial de otimização do processo de produção em larga escala, comum na indústria de plásticos.

 \vspace{\onelineskip}
    
 \noindent
 \textbf{Palavras-chave}: latex. abntex. editoração de texto.
\end{resumo}
